\documentclass[5pt]{ltjsarticle}
\usepackage{luatexja-otf}
\usepackage[]{luatexja,luatexja-fontspec}\usepackage{booktabs}
\usepackage{array}
\usepackage{graphicx}
\usepackage{mathpazo}
\usepackage{amsmath}
\usepackage{amsthm}
\usepackage{amssymb}
\usepackage{amsfonts}
\usepackage[noheadfoot,top=10mm,bottom=10mm,hmargin=10mm]{geometry}
\usepackage{tikz}
\usetikzlibrary{matrix}
\usepackage{pgfcore}
\usepackage{color}
\usepackage{longtable}
\newtheorem{dfn}{Def}
\newtheorem{thm}{Thm}
\newtheorem{cor}{Cor}
\newtheorem{prop}{Prop}
\newtheorem{rk}{remark}
\newtheorem{claim}{claim}
\newtheorem{recall}{recall}
\newtheorem{ques}{Q.}
\date{\today}
\author{俺}
\title{テスト}
\begin{document}\maketitle
\section*{ドリル}

\begin{enumerate}
\item
  次の写像は関手の定義を満たすか満たさないかそれぞれ答えよ:
\item
  関手は同型射を同型射に写すことを示せ
\item
  恒等関手\(1_\mathbf{A}\)は、反対圏への関手\(1_\mathbf{A}: \mathbf{A} \to \mathbf{A^{\mathbf{op}}}\)と見做せることを示せ
\item
  生成した画像はTurtle側から取得しに行くのだ
\item
  これは実質diagramsでTurtleを使ってるのと大差ないのでは?

  \begin{enumerate}
  
  \item
    二つのGHCiを立ち上げなければならないので同じではないが\(\cdots\)
  \end{enumerate}
\item
  moajoさんの後ろの穴をパンパン掘ってドックドックとタネをつけるからパンドック
\end{enumerate}

\subsection{川瀬の金玉潰したい}

\begin{longtable}[]{@{}ll@{}}
\toprule
test & test\tabularnewline
\midrule
\endhead
tst & tst\tabularnewline
\bottomrule
\end{longtable}

\begin{itemize}
\item
  ↑このテーブルの書き方はpipetableとpandocで読んでいるものである。
\item
  literateHaskellの拡張をOnにしたはずだが機能しないのは何故だ?
\item
  MarkdownにinputしたPGFコードはpandocで変換後に正しく読み込まれるのだろうか?
\end{itemize}
\end{document}
